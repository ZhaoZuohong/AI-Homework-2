\documentclass[UTF8,zihao=-4]{ctexart}
\usepackage{graphicx}
\usepackage{geometry}
\usepackage{siunitx}
\usepackage{amsmath}
\usepackage{amsfonts}
\usepackage{amssymb}
\usepackage{listings}
\usepackage{forest}
\usepackage{tikz}
\lstset{
	language=C++,
	breaklines,
	tabsize=4,
	basicstyle=\ttfamily \small
}
\geometry{a4paper,centering,scale=0.8}
\title{\heiti 人工智能\quad 第二次作业}
\author{PB17000005\quad \CJKfontspec{AR PL UKai CN} 赵作竑}
\date{\kaishu \today}
\begin{document}
	\maketitle
	\begin{itemize}
		\item[4.1] 具体求解的过程请见图\ref{q1begin} 到图\ref{q1end} 。
		\begin{figure}[htbp]
			\centering
			\begin{forest}
				[{Lugoj: $244=0+244$}]
			\end{forest}
			\caption{初始状态}
			\label{q1begin}
		\end{figure} 
		\begin{figure}[htbp]
			\centering
			\begin{forest}
				[{Lugoj}
					[{Timisoara: $440=111+329$}]
					[{\underline{Mehadia: $311=70+241$}}]
				]
			\end{forest}
			\caption{扩展Lugoj之后}
		\end{figure} 
		\begin{figure}[htbp]
			\centering
			\begin{forest}
				[{Lugoj}
					[{Timisoara: $440=111+329$}]
					[{Mehadia}
						[{\underline{Dobreta: $387=145+242$}}]
					]
				]
			\end{forest}
			\caption{扩展Mehadia之后}
		\end{figure} 
		\begin{figure}[htbp]
			\centering
			\begin{forest}
				[{Lugoj}
					[{Timisoara: $440=111+329$}]
					[{Mehadia}
						[{Dobreta}
							[{\underline{Craiova: $425=265+160$}}]
						]
					]
				]
			\end{forest}
			\caption{扩展Dobreta之后}
		\end{figure} 
		\begin{figure}[htbp]
			\centering
			\begin{forest}
				[{Lugoj}
					[{\underline{Timisoara: $440=111+329$}}]
					[{Mehadia}
						[{Dobreta}
							[{Craiova}
								[{Rimnicu Vilcea: $604=411+193$}]
								[{Pitesti: $500=403+98$}]
							]
						]
					]
				]
			\end{forest}
			\caption{扩展Craiova之后}
		\end{figure}
		\begin{figure}[htbp]
			\centering
			\begin{forest}
				[{Lugoj}
					[{Timisoara: $440=111+329$}
						[{Arad: $554=188+366$}]
					]
					[{Mehadia}
						[{Dobreta}
							[{Craiova}
								[{Rimnicu Vilcea: $604=411+193$}]
								[{\underline{Pitesti: $500=403+98$}}]
							]
						]
					]
				]
			\end{forest}
			\caption{扩展Timisoara之后}
		\end{figure}
		\begin{figure}[htbp]
			\centering
			\begin{forest}
				[{Lugoj}
					[{Timisoara: $440=111+329$}
						[{Arad: $554=188+366$}]
					]
					[{Mehadia}
						[{Dobreta}
							[{Craiova}
								[{Rimnicu Vilcea: $604=411+193$}]
								[{Pitesti: $500=403+98$}
									[{\underline{Bucharset: $504=504+0$}}]
								]
							]
						]
					]
				]
			\end{forest}
			\caption{扩展Pitesti之后}
			\label{q1end}
		\end{figure}
		\item[4.2] 当$w=0$时,$f(n)=2g(n)$,寻找最小$f(n)$也就是寻找最小的$g(n)$,这个算法是代价一致搜索算法。当$w=1$时,$f(n)=g(n)+h(n)$,这个算法是$A*$搜索。当$w=2$时,$f(n)=2h(n)$,$f(n)$的大小完全由$h(n)$决定,这个算法变成了贪婪最佳优先搜索算法。当$w=1$时,也就是$A*$算法,是最优的。
		\pagebreak 
		\item[4.6] 假设目标状态与起始状态分别如图\ref{begin-and-target} 所示。
		\begin{figure}[htbp]
			\centering
			\begin{tabular}{|c|c|c|}
				\hline
				8 & 7 & 6 \\ \hline
				5 & 4 & 3 \\ \hline
				2 & 1 &   \\ \hline
			\end{tabular}
			\hspace{1.2cm}
			\begin{tabular}{|c|c|c|}
				\hline
				7 & 6 & 3 \\ \hline
				8 &   & 1 \\ \hline
				5 & 4 & 2 \\ \hline
			\end{tabular}
			\caption{左:目标状态;右:起始状态}
			\label{begin-and-target}
		\end{figure}
		根据题意,要求启发式函数有时会“高估”。假设我们选定的启发式函数为:编号为5-8的滑块的曼哈顿距离之和的三倍:
		\begin{equation*}
			g(n) = 3\times \sum_{i=5}^8 d_{\text{Manhattan}}(i)
		\end{equation*}
		当编号为5-8的滑块的曼哈顿距离较大,而编号为1-4的滑块的曼哈顿距离较小的时候,这一估计是偏大的。在上述情况下,使用$A*$算法,得到的路径长度是16步,而最短只需要10步。

		这个``错误''的算法的代码:
		\begin{lstlisting}
#include <cstdlib>
#include <cstring>
#include <iostream>
#include <queue>
#include <set>
#include <vector>

using namespace std;

typedef struct {
    int board[3][3];
    int f; // f(n) = g(n) + h(n)
    int g;
    int h;
    int zeroI;
    int zeroJ;
    int hash;
} unit;

const int target[3][3] = {
    { 8, 7, 6 },
    { 5, 4, 3 },
    { 2, 1, 0 }
};

const int TargetI[9] = { 2, 2, 2, 1, 1, 1, 0, 0, 0 };
const int TargetJ[9] = { 2, 1, 0, 2, 1, 0, 2, 1, 0 };

struct cmp {
    bool operator()(unit a, unit b)
    {
        return a.f > b.f;
    }
};

priority_queue<unit, vector<unit>, cmp> pq;
set<int> visited;

int UnitHash(unit* u)
{
    int* arr = u->board[0];
    int hash = 0;
    for (int i = 0; i < 9; ++i) {
        hash = hash * 9 + arr[i];
    }
    return hash;
}

int Manhattan(unit* u)
{
    int(*arr)[3] = u->board;
    int result = 0;
    for (int i = 0; i < 3; ++i) {
        for (int j = 0; j < 3; ++j) {
            int n = arr[i][j];
            if (n <= 4) {
                continue;
            }
            result += 3 * abs(TargetJ[n] - j);
            result += 3 * abs(TargetI[n] - i);
        }
    }
    return result;
}

void UpdateUnit(unit* u)
{
    u->g += 1;
    u->h = Manhattan(u);
    u->f = u->g + u->h;
    u->hash = UnitHash(u);
}

bool Reached(unit * u)
{
    if (memcmp(u->board, target, sizeof(target)) == 0) {
        return true;
    } else {
        return false;
    }
}

int A_Star(void)
{
    while (!pq.empty()) {
        unit v = pq.top();
        pq.pop();
        if (v.zeroI != 0) {
            unit u = v;
            u.board[u.zeroI][u.zeroJ] = u.board[u.zeroI - 1][u.zeroJ];
            u.zeroI -= 1;
            u.board[u.zeroI][u.zeroJ] = 0;
            UpdateUnit(&u);
            if (Reached(&u)) {
                return u.g;
            }
            if (visited.find(u.hash) == visited.end()) {
                pq.push(u);
                visited.insert(u.hash);
            }
        }
        if (v.zeroJ != 0) {
            unit u = v;
            u.board[u.zeroI][u.zeroJ] = u.board[u.zeroI][u.zeroJ - 1];
            u.zeroJ -= 1;
            u.board[u.zeroI][u.zeroJ] = 0;
            UpdateUnit(&u);
            if (Reached(&u)) {
                return u.g;
            }
            if (visited.find(u.hash) == visited.end()) {
                pq.push(u);
                visited.insert(u.hash);
            }
        }
        if (v.zeroI != 2) {
            unit u = v;
            u.board[u.zeroI][u.zeroJ] = u.board[u.zeroI + 1][u.zeroJ];
            u.zeroI += 1;
            u.board[u.zeroI][u.zeroJ] = 0;
            UpdateUnit(&u);
            if (Reached(&u)) {
                return u.g;
            }
            if (visited.find(u.hash) == visited.end()) {
                pq.push(u);
                visited.insert(u.hash);
            }
        }
        if (v.zeroJ != 2) {
            unit u = v;
            u.board[u.zeroI][u.zeroJ] = u.board[u.zeroI][u.zeroJ + 1];
            u.zeroJ += 1;
            u.board[u.zeroI][u.zeroJ] = 0;
            UpdateUnit(&u);
            if (Reached(&u)) {
                return u.g;
            }
            if (visited.find(u.hash) == visited.end()) {
                pq.push(u);
                visited.insert(u.hash);
            }
        }
    }
    return 0;
}

int main()
{
    unit u;
    for (int i = 0; i < 3; ++i) {
        for (int j = 0; j < 3; ++j) {
            cin >> u.board[i][j];
        }
    }
    u.g = 0;
    u.h = Manhattan(&u);
    u.f = u.h;
    for (int i = 0; i < 3; ++i) {
        for (int j = 0; j < 3; ++j) {
            if (u.board[i][j] == 0) {
                u.zeroI = i;
                u.zeroJ = j;
            }
        }
    }
    u.hash = UnitHash(&u);
    pq.push(u);
    visited.insert(u.hash);
    cout << A_Star() << endl;
    return 0;
}
		\end{lstlisting}
		如果$h$被高估的部分从来不超过$c$,最差的情况下,启发式对最优解$s_1$高估了$c$,而对另一个解$s_2$没有高估,此时若$A*$算法选择了$s_2$,必然说明$A*$算法对$s_2$的估计值不大于$s_1$的估计值,有
		\begin{equation*}
			s_2 \leq s_1 + c
		\end{equation*}
		从而
		\begin{equation*}
			s_2 - s_1 \leq c
		\end{equation*}
		也就是说,$A*$算法返回的解的耗散比最优解多出的部分也不超过$c$。
		\item[4.7] 设启发式$h(n)$是一致的,也就是说,对于每个节点$n$和通过任何行动$a$生成的$n$的每个后继节点$n'$,有:
		\begin{equation*}
			h(n) \leq c(n, a, n') + h(n')
		\end{equation*} 
		也就是说,
		\begin{equation*}
			h(n) - h(n') \leq c(n, a, n')
		\end{equation*}
		上式的左边表示从节点$n$到它的后继节点节点$n'$的估计耗散值,右边表示从节点$n$到$n'$的实际最小耗散值。设节点$n'$通过行动$a'$到达的后继节点为$n''$,节点$n''$通过行动$a''$到达的后继节点为$n^{(3)}$,等等,直到到达目标$n^{(t)}$,我们都有:
		\begin{align*}
			h(n) - h(n') & \leq c(n, a, n') \\
			h(n') - h(n'') & \leq c(n', a', n'') \\
			h(n'') - h(n^{(3)}) & \leq c(n'', a'', n^{(3)}) \\
			\cdots & \cdots \cdots \\
			h(n^{(t - 1)}) - h(n^{t}) & \leq c(n^{t - 1}, a^{(t - 1)}, n^t)
		\end{align*}
		两边分别相加,得到
		\begin{equation*}
			h(n) - h(n^t) \leq c(n,A,n^t)
		\end{equation*}
		也就是说,$h(n)$不会过高估计到达目标的耗散,即启发式$h(n)$是可采纳的。

		接下来,对于罗马尼亚最短路径问题构造一个启发式$l(n)$。
		设从Bucharset到城市$n$的直线距离是$l(n)$,
		那么我们的启发式函数定义为
		$$
			h(n)\triangleq \mu \cdot l(n)
		$$
		其中$\mu$是一个随机因子,取值范围是$0< \mu <1$,且每条路径对应的$\mu$都是不同的。
		由于$\mu l(n) < h(n)$,而$h(n)$从来不会高估,所以$l(n)$亦不会高估距离,$l(n)$是可采纳的。
		同时,虽然$h(n) \leq c(n,a,n') + h(n')$,但是由于$\mu _1$和$\mu _2$之间关系是随机的,
		所以$\mu _1 h(n)$与$c(n,a,n')+\mu _2 h(n')$的大小无法判断,也就是存在$l(n)>c(n,a,n')+l(n')$的情况,
		因此这个启发式尽管是可采纳的,但不是一致的。
		\item[5.6] 按照英文版教材课后习题的要求,我们综合前向检验、MRV与最少约束值启发式,进行回溯搜索。
		
		我们设从个位到十位的进位为$A$,从十位到百位的进位为$B$。约束条件如下:
		\begin{equation*}
			\begin{cases}
				F \neq 0 \\
				T \neq 0 \\
				O + O = 10 \times A + R \\
				W + W + A = 10 \times B + U \\
				T + T + B = 10 \times F + O
			\end{cases}
		\end{equation*}
		进行预处理,得到各个变量可能取值的个数分别为:
		\begin{center}
			\begin{tabular}{|c|c|c|c|c|c|c|c|}
				\hline
				F & A & B & T & R & O & U & W \\ \hline
				1 & 2 & 2 & 5 & 5 & 10 & 10 & 10 \\ \hline
			\end{tabular}
		\end{center}
		按照度启发式的策略,首先考虑$F$唯一的情况:
		\begin{center}
			\begin{forest}
				[{$F=0$}]
			\end{forest}
		\end{center}
		接下来考虑$A$。当$A=0$和$A=1$时,$U,O$的取值个数都是一样的,所以不妨令$A=0$:
		\begin{center}
			\begin{forest}
				[{$F=1$}
					[{$A=0$}]
				]
			\end{forest}
		\end{center}
		进行前向检验,各个变量的值域也在不断更新。各个变量可能取值的数量变为:
		\begin{center}
			\begin{tabular}{|c|c|c|c|c|c|}
				\hline
				B & T & R & O & U & W \\ \hline
				2 & 5 & 5 & 5 & 5 & 10 \\ \hline
			\end{tabular}
		\end{center}
		接下来考虑$B$。当$B=0$时,$O$的取值可能比$B=1$时更多,所以令$B=0$:
		\begin{center}
			\begin{forest}
				[{$F=1$}
					[{$A=0$}
						[{$B=0$}]
					]
				]
			\end{forest}
		\end{center}
		值域变为:
		\begin{center}
			\begin{tabular}{|c|c|c|c|c|}
				\hline
				O & R & T & U & W \\ \hline
				3 & 5 & 5 & 5 & 5 \\ \hline
			\end{tabular}
		\end{center}
		此时,考虑$O$的取值。对于每一个$O$可能的取值,$R$和$T$都是唯一的。
		当$O=0$时$R=0$,然而由于不同字母不能代表相同数字,所以不行;
		当$O=2$时$R=4,T=6$。由于不同字母不能代表相同数字,但是这时对于所有可能的$W$,都不满足要求,所以回溯。
		当$O=4$时$T=7,R=8$。此时$W=3,U=6$即为一个解:
		\begin{center}
			\begin{forest}
				[{$F=1$}
					[{$A=0$}
						[{$B=0$}
							[{\sout{$O=0$}}]
							[{\sout{$O=2$}}]
							[{$O=4$}]
						]
					]
				]
			\end{forest}
		\end{center}
		结果是:
		\begin{equation*}
			\begin{cases}
				T = 7 \\
				W = 3 \\
				O = 4 \\
				F = 1 \\
				U = 6 \\
				R = 8
			\end{cases}
		\end{equation*}
		\begin{center}
			\begin{tabular}{cccc}
				& 7 & 3 & 4 \\
				+ & 7 & 3 & 4 \\ \hline
				1 & 4 & 6 & 8
			\end{tabular}
		\end{center}
		\item[6.11] 用$R,G,B$分别代表红色,绿色和蓝色。从弧WA-NT开始考虑,则$D_{\text{NT}}=\{G,B\}$,
		考虑弧WA-SA,得到$D_{\text{SA}}=\{G,B\}$。考虑弧SA-V,则$D_{\text{SA}}=\{G\}$。
		观察弧NT-SA,得到$D_{\text{NT}}=\{B\}$。继续看弧NT-Q和弧SA-Q,可以看出$D_{\text{Q}}=\{R\}$。

		最后,我们来看与NSW相连的三个弧,就会发现,NSW没有可以选择的颜色,也就是说部分赋值$\{\text{WA}=R,\text{V}=B\}$是不相容的。
		\item[6.12] 假设树每层最多有$b$个分支,最多有$d$层,值域里最多有$c$种取值,则每一个节点向下的弧中最坏有$b\cdot c$次运算。
		第一层最坏为$bc$次,第二层最坏为$b^2\cdot c$次,以此类推,第$d$层最坏为$b^d\cdot c$次运算。
		进行求和:
		\begin{equation*}
			\sum_{i=1}^d b^i \cdot c = c \cdot \sum_{i=1}^d b^i = bc\frac{b^d-1}{b-1} = O(b^d \cdot c)
		\end{equation*}
	\end{itemize}
\end{document}